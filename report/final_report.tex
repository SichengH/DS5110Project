%
% DS 5110 (Blue Team) Project Proposal
%
\documentclass[12pt]{article}

%
% Packages
%
\usepackage{amsmath}
\usepackage{enumerate}
\usepackage[utf8]{inputenc}
\usepackage[toc,page]{appendix}

\RequirePackage{graphics}

\usepackage{graphicx}
\graphicspath{ {imgs/} }

%
% Document Settings
%
\setlength{\parskip}{1pc}
\setlength{\parindent}{0pt}
\setlength{\topmargin}{-3pc}
\setlength{\textheight}{9.5in}
\setlength{\oddsidemargin}{0pc}
\setlength{\evensidemargin}{0pc}
\setlength{\textwidth}{6.5in}

\title{\textbf{Flashlight}: Property Assessment Visualization for the City of Boston}
\author{Tyler Brown, Sicheng Hao, Nischal Mahaveer Chand, Sumedh Sankhe}
\date{ }


% START DOCUMENT
\begin{document}

\maketitle

\section*{Summary}

As a new home-buyer, it's easy to find out about your home
but hard to get an understanding of your neighborhood. Flashlight
makes it easier for you to see your potential neighborhood in Boston.
This discrepancy is because current real estate websites emphasize
individual properties rather than individual neighborhoods. Our group
communicates the differences between Boston neighborhoods using an
interactive data visualization called Flashlight.

Our dataset includes Property Assessment history from 2014-2017
\cite{Property49:online} using Boston's Open Data Hub. We enriched the 
property assessment data with coordinates from Open Addresses
\cite{OpenAddr24:online}, and neighborhood boundaries from Zillow
\cite{ZillowNe81:online}. These combined datasets provide unique
insights to new home-buyers in Boston. As open data becomes more
prevalent in cities across the United States, we can scale our insights
and models.

\section*{Methods}

We used methods for collecting, preparing, modeling, and presenting
our data. Each step of the process is detailed here.

\subsection{Data Collection}

We started with Boston's Property Assessment data from 2014-2017
\cite{Property49:online}. This dataset ``[g]ives property, or parcel,
ownership together with value information, which ensures fair assessment
of Boston taxable and non-taxable property of all types and
classifications.''\cite{Property49:online}. We wanted to use this
information because it helps us capture changes in Boston properties
over time. For example, a remodeled property would change it's property
tax assessment value we have this variable available to us.

After starting with the Property Assessment dataset, we brought in
additional datasets to increase the value of our data collection.
Neighborhoods in Boston were not named or geographically demarcated in
the Property Assessment dataset so we brought in Neighborhood Boundaries
from Zillow \cite{ZillowNe81:online} to make this distinction.
Additionally, geographic coordinates for each assessed property's address
were occasionally not coded correctly or included at all for 2017 so we 
had to bring in those values using Open Addresses
\cite{OpenAddr24:online}. Once neighborhood names, boundaries, and missing
coordinates were available, we were able to proceed to data preparation.

\subsection{Data Preparation}

\subsection{Data Modeling}

\subsection{Data Presentation}

\section*{Results}

We had some results.

\section*{Discussion}

Let's discuss what we did.

\section*{Statement of Contributions}

Together everyone achieves more.

\begin{itemize}
\item \textbf{Tyler Brown:}
\item \textbf{Sicheng Hao:}
\item \textbf{Nischal Mahaveer Chand:}
\item \textbf{Sumedh Sankhe:}
\end{itemize}

\bibliography{references} 
\bibliographystyle{ieeetr}

\begin{appendices}

Appendices here.

\end{appendices}

\end{document}
